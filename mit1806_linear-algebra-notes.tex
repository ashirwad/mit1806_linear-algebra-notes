% % % % % % % % % % % % % % % % % % % % % % % % % % % % % % % % % % % % % % % % 
% LaTeX Cheat Sheet von LaTeX4EI                                                                        
%
% @encode:      UTF-8, tabwidth = 4, newline = LF
% @author:      Emanuel Regnath
% @date:                
%
% % % % % % % % % % % % % % % % % % % % % % % % % % % % % % % % % % % % % % % % 


%---------------------------------------%
%                       P R E A M B L E                         %
%~~~~~~~~~~~~~~~~~~~~~~~~~~~~~~~~~~~~~~~%
% arara: indent: {overwrite: yes, silent: yes}
\PassOptionsToPackage{english}{babel}

% Document Class ===============================================================
\documentclass[fs, footer]{latex4ei}

\usepackage{cprotect}
\usepackage{hyperref}
\usepackage{scientific}

% My packages
\usepackage{makecell}

% language
\selectlanguage{english}

%colors
\colorlet{sectioncolor}{tum_blue_dark2}

% Idee: eigene farbe {blue_dark1}, Zuweisungsfarbe {col:section} oder {col:table}
% F�r \code in sectionbox: neue sectionbox als umgebung mit minipage



% Source Code Listings =========================================================
\usepackage{listings}
\definecolor{listinggray}{gray}{0.9}
\definecolor{lbcolor}{rgb}{0.9,0.9,0.9}
\lstset{
    backgroundcolor=\color{lbcolor},
    basicstyle=\tt,
    tabsize=2,
    language={[LaTeX]TeX},
    %upquote=true,
    aboveskip={0.4\baselineskip},
    belowskip={0.4\baselineskip},
    abovecaptionskip={\baselineskip},
    belowcaptionskip={0\baselineskip},
    columns=fixed,
    showstringspaces=false,
    extendedchars=true,
    linewidth=6.7cm,
    xleftmargin={3pt},
    %framexleftmargin={10pt},
    framexrightmargin={2pt},
    %framextopmargin={9pt},
    %framexbottommargin={9pt},
    %breaklines=true,
    prebreak = \raisebox{0ex}[0ex][0ex]{\ensuremath{\hookleftarrow}},
    frame=single,
    showtabs=false,
    showspaces=false,
    showstringspaces=false,
    identifierstyle=\ttfamily,
    %tagstyle=\bf,
    keywordstyle=\color{tum_blue_dark},
    commentstyle=\color[rgb]{0.133,0.545,0.133},
    stringstyle=\color[rgb]{0.8,  0.1,  0.1},
}

\let\myverb\lstinline
\let\code\lstinline
%\renewcommand{\code}[1]{\lstinline?#1?}

\usepackage{fancyvrb} 
\usepackage{verbdef} 
\DefineShortVerb{\#}



\fancyfoot[R]{Created \today \ at \thistime \qquad \thepage}
\fancyfoot[L]{Homepage: \href{www.latex4ei.de}{www.latex4ei.de} -- please report misstakes \emph{immediately}.}
\fancyfoot[C]{by Emanuel Regnath, contact \emph{\href{mailto:emanuel.regnath@tum.de}{emanuel.regnath@tum.de}}}


% Define BibTeX command
\def\BibTeX{{\rm B\kern-.05em{\sc i\kern-.025em b}\kern-.08em T\kern-.1667em\lower.7ex\hbox{E}\kern-.125emX}}


% Hyperref
\hypersetup{
        pdfcreator={LaTeX2e},
        pdfborder=0 0 0,
        breaklinks=true,
        bookmarksopen=true,
        bookmarksnumbered=true,
        linkcolor=tum_blue_dark,
        urlcolor=tum_blue_dark,
        citecolor=tum_blue_dark,
        colorlinks=true
}


% make boxes robust for verbatim
\let\oldsectionbox\sectionbox
\outer\def\sectionbox{\icprotect\oldsectionbox}


%---------------------------------------%
%                       LaTeX Cheat Sheet                       %
%~~~~~~~~~~~~~~~~~~~~~~~~~~~~~~~~~~~~~~~%

% DOCUMENT_BEGIN ===============================================================
\begin{document}

% Split in 4 Columns ===========================================================
\begin{multicols*}{4}

% TITLE ========================================================================
\fstitle{MIT 18.06 Notes \\[0.3em] {\normalfont \small Notes from Dr. Strang's course on linear algebra}}


%{\large Principle: �Write clear \& beautiful english with \textrm{\LaTeX}!�}
% options for environments \tabcolsep
% links to package description in headings
% lstlisting, subfigure
% how to include beautiful matlab plots
% overwiev of important packages:
% multicol, rotate, chem


% SECTION ====================================================================================
\section{Geometry of linear equations}
% ============================================================================================

\sectionbox{
        \subsection{Overview}
        This lecture mainly discusses the three visualization techniques--matrix picture, row picture, and column picture--that are typically used for understanding linear equations, and very briefly touches upon the concepts of linear combination and matrix multiplication.  
}

\sectionbox{
        \subsection{Linear system visualization techniques}
        Dr. Strang proposed the following three visual techniques to understand a linear system of equations before solving them:
        
        \begin{tabular}{@{}ll@{}}\trule
                \textrm{Picture type} & \textrm{Explanation} \\ \mrule
                Matrix picture & \makecell[tl]{It's the algebra way to visualize a linear system using \\ the equation $Ax = b$*} \\
                Row picture & \makecell[tl]{It's the picture created from plotting the solutions to \\ each equation (or row of $A$) in the linear system} \\
                Column picture & \makecell[tl]{It's the picture created from plotting the columns of \\ $A$ and their linear combination that yields $b$} \\ \brule
        \end{tabular}
        *$A$ is the coefficient matrix, $x$ is the vector of unknowns, and $b$ is the right-hand side vector \\

        The subsequent sections show the examples of matrix picture, row picture, and column picture for $2 \times 2$ and $3 \times 3$ system of equations.
}

% Ende der Spalten
\end{multicols*}

% Dokumentende
% ======================================================================
\end{document}


