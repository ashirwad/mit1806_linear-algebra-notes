%
\documentclass[../main.tex]{subfiles}

\begin{document}

\onlyinsubfile{\begin{multicols*}{4}}

\section{Multiplication and inverse matrices}

\sectionboxnew{
\subsection{Multiplying matrices A and B by row times column}
The first approach to multiplying matrices A and B is to start with a row of matrix A and take its dot product with the corresponding column of matrix B to get a matrix entry of the resulting matrix C.
\resizebox{\hsize}{!}{
\[
    \renewcommand\arraystretch{1.6}
    \underbracket[0pt][0pt]{
        \begin{bmatrix}
                   & \vdots &        &        \\
            a_{31} & a_{32} & \ldots & a_{3n} \\
                   & \vdots &        &
        \end{bmatrix}
    }_{\mathstrut \underset{m \times n}{A}}
    \underbracket[0pt][0pt]{
        \begin{bmatrix}
                   & b_{14} &        \\
            \ldots & b_{24} & \ldots \\
                   & \vdots &        \\
                   & b_{n4} &
        \end{bmatrix}
    }_{\mathstrut \underset{n \times p}{B}}
    =
    \underbracket[0pt][0pt]{
        \begin{bmatrix}
                   & \vdots &        \\
            \ldots & c_{34} & \ldots \\
                   & \vdots &
        \end{bmatrix}
    }_{\mathstrut \underset{m \times p}{C}}
\]
}
In the illustration, the $c_{34}$ element, i.e., the element located at row 3 and column 4 of matrix C  is obtained by taking the dot product of the 3rd row of matrix A with the 4th column of matrix B, i.e., $\sum_{k=1}^{n} a_{3 k} b_{k 4}$.
}

\sectionboxnew{
\subsection{Multiplying matrices A and B by columns}
The second approach of multiplying matrices A and B is to start with a column of matrix B and use its elements to linearly combine the columns of matrix A to obtain the corresponding column of the resulting matrix C.
\[
    \renewcommand\arraystretch{1.6}
    \underbracket[0pt][0pt]{
        \begin{bmatrix}
            % use of kern and vline is taken from the SO answer: https://tex.stackexchange.com/a/59614/180993
            \kern.6em\vline & \kern.2em\vline\kern.2em & \vline\kern.6em \\
            \kern.6em\vline & \kern.2em\vline\kern.2em & \vline\kern.6em \\
            \kern.6em\vline & \kern.2em\vline\kern.2em & \vline\kern.6em
        \end{bmatrix}
    }_{\mathstrut \underset{m \times n}{A}}
    \underbracket[0pt][0pt]{
        \begin{bmatrix}
                   & \kern.2em\vline\kern.2em &       \\
            \ldots & col \ j                  & \dots \\
                   & \kern.2em\vline\kern.2em &
        \end{bmatrix}
    }_{\mathstrut \underset{n \times p}{B}}
    =
    \underbracket[0pt][0pt]{
        \begin{bmatrix}
                   & \kern.2em\vline\kern.2em &        \\
            \ldots & A \times col \ j         & \ldots \\
                   & \kern.2em\vline\kern.2em &        \\
        \end{bmatrix}
    }_{\mathstrut \underset{m \times p}{C}}
\]
}

\sectionboxnew{
\subsection{Multiplying matrices A and B by rows}
The third approach of multiplying matrices A and B is to start with a row of matrix A and use its elements to linearly combine the rows of matrix B to obtain the corresponding row of the resulting matrix C.
\[
    \renewcommand\arraystretch{1.6}
    \underbracket[0pt][0pt]{
        \begin{bmatrix}
            \vdots                                                     \\
            % rule definition is adapted from the SO answer: https://tex.stackexchange.com/a/59614/180993
            \rule[.5ex]{1.5em}{0.4pt}row \ i \rule[.5ex]{1.5em}{0.4pt} \\
            \vdots
        \end{bmatrix}
    }_{\mathstrut \underset{m \times n}{A}}
    \underbracket[0pt][0pt]{
        \begin{bmatrix}
            \rule[.5ex]{5em}{0.4pt} \\
            \rule[.5ex]{5em}{0.4pt} \\
            \rule[.5ex]{5em}{0.4pt} \\
            \rule[.5ex]{5em}{0.4pt}
        \end{bmatrix}
    }_{\mathstrut \underset{n \times p}{B}}
    =
    \underbracket[0pt][0pt]{
        \begin{bmatrix}
            \vdots                                                              \\
            \rule[.5ex]{1.5em}{0.4pt}row \ i \times B \rule[.5ex]{1.5em}{0.4pt} \\
            \vdots
        \end{bmatrix}
    }_{\mathstrut \underset{m \times p}{C}}
\]
}

\sectionboxnew{
\subsection{Multiplying matrices A and B by column times row}
The fourth approach of multiplying matrices A and B is to generate a bunch of matrices by multiplying the columns of matrix A with the corresponding rows of matrix B and then add them together.
\[
    \renewcommand\arraystretch{1.6}
    \equalto{
    \underbracket[0pt][0pt]{
        \begin{bmatrix}
            \kern.6em\vline &        & \kern.2em\vline\kern.2em &        & \kern.6em\vline \\
            col \ 1         & \ldots & col \ i                  & \ldots & col \ n         \\
            \kern.6em\vline &        & \kern.2em\vline\kern.2em &        & \kern.6em\vline
        \end{bmatrix}
    }_{\mathstrut \underset{m \times n}{A}}
    \underbracket[0pt][0pt]{
        \begin{bmatrix}
            \rule[.5ex]{1.5em}{0.4pt}row \ 1 \rule[.5ex]{1.5em}{0.4pt} \\
            \vdots                                                     \\
            \rule[.5ex]{1.5em}{0.4pt}row \ j \rule[.5ex]{1.5em}{0.4pt} \\
            \vdots                                                     \\
            \rule[.5ex]{1.5em}{0.4pt}row \ n \rule[.5ex]{1.5em}{0.4pt}
        \end{bmatrix}
    }_{\mathstrut \underset{n \times p}{B}}}
    {
    \underbracket[0pt][0pt]{
        \begin{bmatrix}
            \horzbar & \horzbar & \horzbar \\
            \horzbar & \horzbar & \horzbar \\
            \horzbar & \horzbar & \horzbar
        \end{bmatrix}
    }_{\mathstrut \underset{m \times p}{C_1}}
    +
    \ldots
    +
    \underbracket[0pt][0pt]{
        \begin{bmatrix}
            \horzbar & \horzbar & \horzbar \\
            \horzbar & \horzbar & \horzbar \\
            \horzbar & \horzbar & \horzbar
        \end{bmatrix}
    }_{\mathstrut \underset{m \times p}{C_i}}
    +
    \ldots
    +
    \underbracket[0pt][0pt]{
        \begin{bmatrix}
            \horzbar & \horzbar & \horzbar \\
            \horzbar & \horzbar & \horzbar \\
            \horzbar & \horzbar & \horzbar
        \end{bmatrix}
    }_{\mathstrut \underset{m \times p}{C_n}}}
\]
Let's illustrate this matrix multiplication approach with the following example:
\begingroup
\allowdisplaybreaks
\begin{align*}
    \begin{bmatrix}
        2 & 7 \\
        3 & 8 \\
        4 & 9
    \end{bmatrix}
    \begin{bmatrix}
        1 & 6 \\
        0 & 0
    \end{bmatrix}
     & =
    \begin{bmatrix}
        2 \\
        3 \\
        4
    \end{bmatrix}
    \begin{bmatrix}
        1 & 6
    \end{bmatrix}
    +
    \begin{bmatrix}
        7 \\
        8 \\
        9
    \end{bmatrix}
    \begin{bmatrix}
        0 & 0
    \end{bmatrix} \\
     & =
    \begin{bmatrix}
        2 \times 1 & 2 \times 6 \\
        3 \times 1 & 3 \times 6 \\
        4 \times 1 & 4 \times 6
    \end{bmatrix}
    +
    \begin{bmatrix}
        7 \times 0 & 7 \times 0 \\
        8 \times 0 & 8 \times 0 \\
        9 \times 0 & 9 \times 0
    \end{bmatrix} \\
     & =
    \begin{bmatrix}
        2 & 12 \\
        3 & 18 \\
        4 & 24
    \end{bmatrix}
    +
    \begin{bmatrix}
        0 & 0 \\
        0 & 0 \\
        0 & 0
    \end{bmatrix}
    =
    \begin{bmatrix}
        2 & 12 \\
        3 & 18 \\
        4 & 24
    \end{bmatrix}
\end{align*}
\endgroup
The resulting matrix $\begin{bsmallmatrix} 2 & 12 \\ 3 & 18 \\ 4 & 24 \end{bsmallmatrix}$  is a special matrix, where:
\begin{itemize}
    \item all of the rows lie on the line through the vector $\begin{bsmallmatrix} 1 \\ 6 \end{bsmallmatrix}$, and
    \item  all of the columns lie on the line through the vector $\begin{bsmallmatrix} 2 \\ 3 \\ 4 \end{bsmallmatrix}$
\end{itemize}
In other words, the row space and the column space of this matrix is a line.
}

\sectionboxnew{
    \subsection{Multiplying matrices A and B by blocks}
    The fifth approach of multiplying matrices A and B is to cut the matrices into blocks and then multiply block rows of A with block columns of B to get the corresponding blocks of the resulting matrix C.

    \resizebox{\hsize}{!}{
        $
            \renewcommand\arraystretch{1.6}
            % adapted from the SO answer: https://tex.stackexchange.com/a/230505/180993
            \sbox0{$\begin{matrix}1 & 0 \\ 0 & 1\end{matrix}$}
            \underbracket[0pt][0pt]{
                \left[
                    \begin{array}{c|c}
                        \vphantom{\usebox{0}}\makebox[\wd0]{\large $A_1$} & \makebox[\wd0]{\large $A_2$} \\
                        \hline
                        \vphantom{\usebox{0}}\makebox[\wd0]{\large $A_3$} & \makebox[\wd0]{\large $A_4$}
                    \end{array}
                    \right]
            }_{\mathstrut \underset{m \times n}{A}}
            \underbracket[0pt][0pt]{
                \left[
                    \begin{array}{c|c}
                        \vphantom{\usebox{0}}\makebox[\wd0]{\large $B_1$} & \makebox[\wd0]{\large $B_2$} \\
                        \hline
                        \vphantom{\usebox{0}}\makebox[\wd0]{\large $B_3$} & \makebox[\wd0]{\large $B_4$}
                    \end{array}
                    \right]
            }_{\mathstrut \underset{n \times p}{B}}
            =
            \underbracket[0pt][0pt]{
                \left[
                    \begin{array}{c|c}
                        \vphantom{\usebox{0}}\makebox[\wd0]{\large $C_1$} & \makebox[\wd0]{\large $C_2$} \\
                        \hline
                        \vphantom{\usebox{0}}\makebox[\wd0]{\large $C_3$} & \makebox[\wd0]{\large $C_4$}
                    \end{array}
                    \right]
            }_{\mathstrut \underset{m \times p}{C}}
        $
    }
    The first block, $C_1$, is the dot product of the first block row of matrix A and the first block column of matrix B, i.e.,  $A_1\times B_1 + A_2 \times B_3$, and other blocks are obtained similarly.
}

\onlyinsubfile{\end{multicols*}}

\end{document}
