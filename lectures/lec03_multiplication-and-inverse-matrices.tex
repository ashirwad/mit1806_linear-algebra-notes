%
\documentclass[../main.tex]{subfiles}

\begin{document}

\onlyinsubfile{\begin{multicols*}{4}}

\section{Multiplication and inverse matrices}

\sectionboxnew{
\subsection{Multiplying matrices A and B by columns}
The second approach of multiplying matrices A and B is to start with a column of matrix B and use its elements to linearly combine the columns of matrix A to obtain the corresponding column of the resulting matrix C.
\[
    \renewcommand\arraystretch{1.6}
    \underbracket[0pt][0pt]{
        \begin{bmatrix}
            % use of kern and vline is taken from the SO answer: https://tex.stackexchange.com/a/59614/180993
            \kern.6em\vline & \kern.2em\vline\kern.2em & \vline\kern.6em \\
            \kern.6em\vline & \kern.2em\vline\kern.2em & \vline\kern.6em \\
            \kern.6em\vline & \kern.2em\vline\kern.2em & \vline\kern.6em
        \end{bmatrix}
    }_{\mathstrut \underset{m \times n}{A}}
    \underbracket[0pt][0pt]{
        \begin{bmatrix}
                   & \kern.2em\vline\kern.2em &       \\
            \ldots & col \ j                  & \dots \\
                   & \kern.2em\vline\kern.2em &
        \end{bmatrix}
    }_{\mathstrut \underset{n \times p}{B}}
    =
    \underbracket[0pt][0pt]{
        \begin{bmatrix}
                   & \kern.2em\vline\kern.2em &        \\
            \ldots & A \times col \ j         & \ldots \\
                   & \kern.2em\vline\kern.2em &        \\
        \end{bmatrix}
    }_{\mathstrut \underset{m \times p}{C}}
\]
}

\sectionboxnew{
\subsection{Multiplying matrices A and B by rows}
The third approach of multiplying matrices A and B is to start with a row of matrix A and use its elements to linearly combine the rows of matrix B to obtain the corresponding row of the resulting matrix C.
\[
    \renewcommand\arraystretch{1.6}
    \underbracket[0pt][0pt]{
        \begin{bmatrix}
            \vdots                                                     \\
            % rule definition is adapted from the SO answer: https://tex.stackexchange.com/a/59614/180993
            \rule[.5ex]{1.5em}{0.4pt}row \ i \rule[.5ex]{1.5em}{0.4pt} \\
            \vdots
        \end{bmatrix}
    }_{\mathstrut \underset{m \times n}{A}}
    \underbracket[0pt][0pt]{
        \begin{bmatrix}
            \rule[.5ex]{5em}{0.4pt} \\
            \rule[.5ex]{5em}{0.4pt} \\
            \rule[.5ex]{5em}{0.4pt} \\
            \rule[.5ex]{5em}{0.4pt}
        \end{bmatrix}
    }_{\mathstrut \underset{n \times p}{B}}
    =
    \underbracket[0pt][0pt]{
        \begin{bmatrix}
            \vdots                                                              \\
            \rule[.5ex]{1.5em}{0.4pt}row \ i \times B \rule[.5ex]{1.5em}{0.4pt} \\
            \vdots
        \end{bmatrix}
    }_{\mathstrut \underset{m \times p}{C}}
\]
}

\sectionboxnew{
    \subsection{Multiplying matrices A and B by blocks}
    The fifth approach of multiplying matrices A and B is to cut the matrices into blocks and then multiply block rows of A with block columns of B to get the corresponding blocks of the resulting matrix C.

    \resizebox{\hsize}{!}{
        \[
            \renewcommand\arraystretch{1.6}
            % adapted from the SO answer: https://tex.stackexchange.com/a/230505/180993
            \sbox0{$\begin{matrix}1 & 0 \\ 0 & 1\end{matrix}$}
            \underbracket[0pt][0pt]{
                \left[
                    \begin{array}{c|c}
                        \vphantom{\usebox{0}}\makebox[\wd0]{\large $A_1$} & \makebox[\wd0]{\large $A_2$} \\
                        \hline
                        \vphantom{\usebox{0}}\makebox[\wd0]{\large $A_3$} & \makebox[\wd0]{\large $A_4$}
                    \end{array}
                    \right]
            }_{\mathstrut \underset{m \times n}{A}}
            \underbracket[0pt][0pt]{
                \left[
                    \begin{array}{c|c}
                        \vphantom{\usebox{0}}\makebox[\wd0]{\large $B_1$} & \makebox[\wd0]{\large $B_2$} \\
                        \hline
                        \vphantom{\usebox{0}}\makebox[\wd0]{\large $B_3$} & \makebox[\wd0]{\large $B_4$}
                    \end{array}
                    \right]
            }_{\mathstrut \underset{n \times p}{B}}
            =
            \underbracket[0pt][0pt]{
                \left[
                    \begin{array}{c|c}
                        \vphantom{\usebox{0}}\makebox[\wd0]{\large $C_1$} & \makebox[\wd0]{\large $C_2$} \\
                        \hline
                        \vphantom{\usebox{0}}\makebox[\wd0]{\large $C_3$} & \makebox[\wd0]{\large $C_4$}
                    \end{array}
                    \right]
            }_{\mathstrut \underset{m \times p}{C}}
        \]
    }
    The first block, $C_1$, is the dot product of the first block row of matrix A and the first block column of matrix B, i.e.,  $A_1\times B_1 + A_2 \times B_3$, and other blocks are obtained similarly.
}

\onlyinsubfile{\end{multicols*}}

\end{document}
