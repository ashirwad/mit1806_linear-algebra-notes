%
\documentclass[../main.tex]{subfiles}

\begin{document}

\begin{multicols*}{4}

\section{Elimination with matrices}

\sectionboxnew{
\subsection{Gauss elimination}
Gauss elimination is a systematic way for solving any system of linear equations. The rule simply involves selecting a pivot in each row and cleaning everything below it through row reduction. \\

The elimination generally succeeds normally, but may encounter either of the two other situations besides normal success:

\begin{tabular}{@{}ll@{}} \trule
\textrm{Situation} & \textrm{Explanation} \\ \mrule
Normal success & \makecell[tl]{Gauss elimination succeeds normally without any \\ row exchanges} \\
Temporary failure & Gauss elimination succeeds with row exchanges \\
Permanent failure & Gauss elimination does not succeed \\ \brule 
\end{tabular} \\

The following sections illustrate each of the three situations encountered by Gauss elimination.
}

\end{multicols*}
\end{document}
