%
\documentclass[../main.tex]{subfiles}

\begin{document}

\onlyinsubfile{\begin{multicols*}{4}}

        \section{Elimination with matrices}

        \sectionboxnew{
            \subsection{Gauss elimination}
            Gauss elimination is a systematic way for solving any system of linear equations. The rule simply involves selecting a pivot in each row and cleaning everything below it through row reduction. \\

            The elimination generally succeeds normally, but may encounter either of the two other situations besides normal success:

            \begin{tabular}{@{}ll@{}} \trule
                \textrm{Situation} & \textrm{Explanation}                                          \\ \mrule
                Normal success     & \makecell[tl]{Gauss elimination succeeds normally without any \\ row exchanges} \\
                Temporary failure  & Gauss elimination succeeds with row exchanges                 \\
                Permanent failure  & Gauss elimination does not succeed                            \\ \brule
            \end{tabular} \\

            The following sections illustrate each of the three situations encountered by Gauss elimination.
        }

        \sectionboxnew{
            \subsubsection{Normal success}
            Let's solve this linear system using Gauss elimination and see it succeed normally:
            \[
                \systeme{x + 2y + z = 2, 3x + 8y + z = 12, 4y + z = 2}
            \]
            \textbf{Step 1:} Start with marking the first pivot at $(1, 1)$ position in the coefficient matrix $A$ and clean off $(2, 1)$ position

            \begin{gmatrix}[b]
                \boxed{1} & 2 & 1 \\
                3 & 8 & 1 \\
                0 & 4 & 1
                \rowops
                \add[-3]{0}{1}
            \end{gmatrix}
            \xrightarrow{E_{21}}
            \begin{gmatrix}[b]
                \boxed{1} & 2 & 1 \\
                0 & 2 & -2 \\
                0 & 4 & 1
            \end{gmatrix} \\

            \textbf{Step 2:} Since everything below the first pivot is now cleaned, mark the second pivot at $(2, 2)$  and clean off $(3, 2)$

            \begin{gmatrix}[b]
                \boxed{1} & 2 & 1 \\
                0 & \boxed{2} & -2 \\
                0 & 4 & 1
                \rowops
                \add[-2]{1}{2}
            \end{gmatrix}
            \xrightarrow{E_{32}}
            \begin{gmatrix}[b]
                \boxed{1} & 2 & 1 \\
                0 & \boxed{2} & -2 \\
                0 & 0 & 5
            \end{gmatrix} \\

            \textbf{Final step:} Since everything below the second pivot is now clean and there is nothing to clean below the third pivot at $(3, 3)$, just mark it and call it the end

            \begin{gmatrix}[b]
                \boxed{1} & 2 & 1 \\
                0 & \boxed{2} & -2 \\
                0 & 0 & \boxed{5}
            \end{gmatrix}
            \right \qquad \text{\parbox{4cm}{
                    \textbf{Fact 1:} This matrix is called an \textit{upper triangular matrix}, usually denoted with \textbf{$U$} \\[0.25cm]
                    \textbf{Fact 2:} The product of the pivots of $U$ gives its \textit{determinant}, so here $det(U) = 10$}
            }
        }

        \sectionboxnew{
            \subsubsection{Temporary failure}
            Let's solve this linear system using Gauss elimination and see it overcome temporary failure:
            \[
                \systeme{x + 2y + z = 2, 3x + 6y + z = 12, 4y + z = 2}
            \]
            \textbf{Step 1:} It is same as the \textit{normal success example}

            \begin{gmatrix}[b]
                \boxed{1} & 2 & 1 \\
                3 & 6 & 1 \\
                0 & 4 & 1
                \rowops
                \add[-3]{0}{1}
            \end{gmatrix}
            \xrightarrow{E_{21}}
            \begin{gmatrix}[b]
                \boxed{1} & 2 & 1 \\
                0 & 0 & -2 \\
                0 & 4& 1
            \end{gmatrix}\\

            \textbf{Step 2:} Similar to the \textit{normal success example}, entries below the first pivot are clean, but now there is an undesirable $0$ at the second pivot position $(2, 2)$. Let's swap it with the row below to get a non-zero entry.

            \begin{gmatrix}[b]
                \boxed{1} & 2 & 1 \\
                0 & \boxed{0} & -2 \\
                0 & 4& 1
                \rowops
                \swap{1}{2}
            \end{gmatrix}
            \xrightarrow{P_{23}}
            \begin{gmatrix}[b]
                \boxed{1} & 2 & 1 \\
                0 & \boxed{4} & 1 \\
                0 & 0 & -2
            \end{gmatrix}\\

            \textbf{Final step:} It is also same as the \textit{normal success example}

            \begin{gmatrix}[b]
                \boxed{1} & 2 & 1 \\
                0 & \boxed{4} & 1 \\
                0 & 0 & \boxed{-2}
            \end{gmatrix}
        }

        \sectionboxnew{
            \subsection{Permanent failure}
            Let's solve this linear system using Gauss elimination and see it fail to escape failure:
            \[
                \systeme{x + 2y + z = 2, 3x + 8y + z = 12, 4y - 4z = 2}
            \]
            \textbf{Step 1:} Same as the \textit{normal success example}

            \begin{gmatrix}[b]
                \boxed{1} & 2 & 1 \\
                3 & 8 & 1 \\
                0 & 4 & -4
                \rowops
                \add[-3]{0}{1}
            \end{gmatrix}
            \xrightarrow{E_{21}}
            \begin{gmatrix}[b]
                \boxed{1} & 2 & 1 \\
                0 & 2 & -2 \\
                0 & 4& -4
            \end{gmatrix}\\

            \textbf{Step 2:} Same as the \textit{normal success example}

            \begin{gmatrix}[b]
                \boxed{1} & 2 & 1 \\
                0 & \boxed{2} & -2 \\
                0 & 4& -4
                \rowops
                \add[-2]{1}{2}
            \end{gmatrix}
            \xrightarrow{E_{32}}
            \begin{gmatrix}[b]
                \boxed{1} & 2 & 1 \\
                0 & \boxed{2} & -2 \\
                0 & 0 & 0
            \end{gmatrix}\\

            \textbf{Final step:} Entries below the first and the second pivots are all clean, which is good, but there is a \mathbf{0} at the pivot position with no row below to exchange it, so call it the end and declare elimination failure.

            \begin{gmatrix}[b]
                \boxed{1} & 2 & 1 \\
                0 & \boxed{2} & -2 \\
                0 & 0 & \boxed{\textcolor{red}{0}}
            \end{gmatrix}\right \qquad
            \text{\parbox{4cm}{
                    \textbf{Fact 1:} As noted before, product of pivot elements of $U$ give its  determinant, so here $det(U) = 0$\\[0.25cm]
                    \textbf{Fact 2:} A square matrix with a 0 determinant value is called \textit{non-invertible} or \textit{singular} matrix}
            }
        }

        \sectionboxnew{
            \subsection{Back substitution}
            Back substitution is a technique used for obtaining the solution to the linear system by working bottom-up with the reduced form of the system, $UX = c$, obtained through Gauss elimination. \\

            Create an augmented matrix for equation ... by tacking on the right-hand-side vector onto the coefficient matrix $A$ and repeat the Gauss elimination steps from \textit{normal success example}: \\

            Step 1:

            \begin{gmatrix}[b]
                1 & 2 & 1 & \BAR & 2 \\
                3 & 8 & 1 & \BAR & 12 \\
                0 & 4 & 1 & \BAR & 2
                \rowops
                \add[-3]{0}{1}
            \end{gmatrix}
            \xrightarrow{E_{21}}
            \begin{gmatrix}[b]
                1 & 2 & 1 & \BAR & 2 \\
                0 & 2 & -2 & \BAR & 6 \\
                0 & 4 & 1 & \BAR & 2
            \end{gmatrix}

            Step 2:

            \begin{gmatrix}[b]
                1 & 2 & 1 & \BAR & 2 \\
                0 & 2 & -2 & \BAR & 6 \\
                0 & 4 & 1 & \BAR & 2
                \rowops
                \add[-2]{1}{2}
            \end{gmatrix}
            \xrightarrow{E_{32}}
            \begin{gmatrix}[b]
                1 & 2 & 1 & \BAR & 2 \\
                0 & 2 & -2 & \BAR & 6 \\
                0 & 0 & 5 & \BAR & -10
            \end{gmatrix}

            The system of equation shown in ... can now be rewritten as:
            \[
                \systeme{x + 2y + z = 2, 2y - 2z = 6, 5z = -10}
            \]
            Use back-substitution to obtain $z = -2$, $y = 1$ and $x = 2$ as the solution to equation ...
        }


        \onlyinsubfile{\end{multicols*}}
\end{document}
