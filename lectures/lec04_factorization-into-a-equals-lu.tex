%
\documentclass[../main.tex]{subfiles}

\begin{document}

\onlyinsubfile{\begin{multicols*}{4}}

\section{Factorization into A equals LU}

\sectionboxnew{
\subsection{Rules of inverses and transposes}
Transpose of a matrix $A$, generally denoted as $A^{T}$, is a matrix obtained by exchanging the rows and columns of matrix $A$. So, a matrix element at $(i, j)$ position in $A$ gets moved to $(j, i)$ position in $A^{T}$. \\

The following are the rules for finding inverses and transposes for a product of matrices: 
\begin{enumerate}
\item Inverse of a product of matrices is equal to the product of the inverses in the reverse order, i.e., 
\[(AB)^{-1} = B^{-1}A^{-1}\]
\item Similarly, transpose of a product of matrices is equal to the product of the transposes in the reverse order, i.e., 
\[(AB)^{T} = B^{T}A^{T}\]
\end{enumerate}
Recall the definition of an inverse: $AA^{-1} = I$, and take the transpose on both sides:
\[
(AA^{-1})^{T} = I^{T} = I
\]
Also, from the rule for the transpose of a product of matrices, $(AA^{-1})^{T}$ can be expressed as:
\[
(AA^{-1})^{T} = (A^{-1})^{T}A^{T}
\]
Comparing the two different expressions for $(AA^{-1})^T$, we get the following:
\[
 (A^{-1})^{T}A^{T} = I \implies \text{$(A^{-1})^{T}$ is the inverse of $A^{T}$}
\] 
\\
So, if we know the inverse of $A$, we can take its transpose to get the inverse of $A^{T}$.
}


\onlyinsubfile{\end{multicols*}}

\end{document}
